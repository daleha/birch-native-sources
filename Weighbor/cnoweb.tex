% c-NO_WEB (a c program pretty printer)
% by Jim Fox, February 1987
% version 1.1, August 10, 1989
% version 1.2, October 30, 1989
%              fixed spacing after ?, :, etc.
% version 1.3, June 28, 1990
%              added OOdent and IIdent, added \|typewriter|
% version 1.4, January 4, 1991
%              allow /* and */ in strings
% version 1.5, December 11, 1991
%              page offset shift for two-sided page print alignment
%              nicer setting of too-long code lines
%              add \listfile and \includefile macros
% version 1.6, March 17, 1992
%              Restore plain-tex meaning of right quote in math
%
%------------------------------------------------------------------
%
% The TeX input file is a c source program.  It is assumed to have
% this structure:
%
%   /* comments ignored by c-no-web
%      \input cnoweb
%      \title{A one line title of the program (optional) }
%      \synopsis{Synopsis of the program (optional) }
%      \section{Name of the first section}
%      rest of comments */
%
% The rest of the file is normal c code.
% Within the title a \\ may be used to separate lines.
%
% Each major section begins with
%   /* \section{name of the section}    .... */
% The section names are printed boldface, marked for reference in
% the page headline, and written to the table of contents.
%
% Each minor section begins with
%   /* \subsection{name of the subsection}    .... */
% The subsection names are written to the table of contents.
%
% A new page can be forced by
%      /* \newpage */
%
% Code and other text containing special characters may be
% typed in text mode if it is bracked by *< ... >*
% e.g. /* *<inactive c-code>* */
% but this may not be used in a macro argument.
%
% You can print in bold by enclosing text in double quotes: \"text".
% You can print in typewriter by enclosing text in vertical bars: \|text|.
% You can print in italics by enclosing text in single quotes: \'text'.
%
% You can include a verbatim listing of a file with \listfile, e.g.,
%      ... \listfile{program.h} ...
% You can include a c file to be formatted with \includefile, e.g.,
%      ... \includefile{program.h} ...
%     
% The file must end with \endc in a comment. e.g.,
%      ... last c code statement ...
%      /* \endc */
%
%-----------------------------------------------------------------
%
% Make sure macros loaded only once
%
\expandafter\ifx\csname cnoweb loaded\endcsname\relax
  \expandafter\let\csname cnoweb loaded\endcsname*
  \else\endinput\fi
%
\catcode`\@=11
%
\message{[c-no-web vers 1.6]}
%
%
% page parameters -- most of these can be changed without harm
% The default setup is for 12 point type on an 8.5 x 11 inch page.
% Two side printing is assumed, 1-inch margins on the inside.
%
\vsize=55pc
\hsize=37pc
\topskip=2pc
\parindent=0pt
\parskip=6pt
\overfullrule=0pt
\newdimen\oddsideshift \oddsideshift=0pc % adjustment to odd, left margins
\newdimen\evensideshift \evensideshift=2pc % adjustment to even, left margins
\newdimen\rightmarginslop \rightmarginslop=3pc % right margin slop in code
\newdimen\blockindent \blockindent=18pt % '{' block indentation
\newdimen\longblockindent \longblockindent=30pt % '(' indentation
\newskip\sectionskip \sectionskip2pc
%
% fonts
%
\font\titles=cmr10 scaled\magstep3     % for titles
%
% normal commentary size is 12 point
%
\font\twelverm=cmr12 % roman text
\font\twelvei=cmmi12 % math italic
\font\twelvesy=cmsy10 scaled\magstep1 % math symbols
\font\twelvebf=cmbx12 % boldface extended
\font\twelvett=cmtt12 % typewriter
\font\twelvesl=cmsl12 % slanted roman
\font\twelveit=cmti12 % text italic
\font\twelveex=cmex10 scaled\magstep1 % math symbols
%
\font\ninerm=cmr9 % roman text
\font\ninei=cmmi9 % math italic
\font\ninesy=cmsy9 % math symbols
%
\def\normaltype{%
 \def\rm{\fam0\twelverm}%
 \textfont0=\twelverm \scriptfont0=\ninerm \scriptscriptfont0=\sevenrm
 \textfont1=\twelvei  \scriptfont1=\ninei  \scriptscriptfont1=\seveni
 \textfont2=\twelvesy \scriptfont2=\ninesy \scriptscriptfont2=\sevensy
 \textfont3=\twelveex \scriptfont3=\twelveex \scriptscriptfont3=\twelveex
 \textfont\itfam=\twelveit \def\it{\fam\itfam\twelveit}%
 \textfont\slfam=\twelvesl \def\sl{\fam\slfam\twelvesl}%
 \textfont\bffam=\twelvebf \def\bf{\fam\bffam\twelvebf}%
 \textfont\ttfam=\twelvett \def\tt{\fam\ttfam\twelvett}%
 \normalbaselineskip 14\p@
 \setbox\strutbox=\hbox{\vrule height10\p@ depth4\p@
      width0\p@}%
 \normalbaselines\rm}
\normaltype
%
% normal code size is 11 point
%
\font\elevenrm=cmr10  scaled\magstephalf % roman text
\font\eleveni=cmmi10  scaled\magstephalf % math italic
\font\elevensy=cmsy10  scaled\magstephalf % math symbols
\font\elevenbf=cmbx10  scaled\magstephalf % boldface extended
\font\eleventt=cmtt10  scaled\magstephalf % typewriter
\font\elevensl=cmsl10  scaled\magstephalf % slanted roman
\font\elevenit=cmti10  scaled\magstephalf % text italic
%
\def\codetype{%
 \def\rm{\fam0\elevenrm}%
 \textfont0=\elevenrm \scriptfont0=\sevenrm \scriptscriptfont0=\fiverm
 \textfont1=\eleveni  \scriptfont1=\seveni  \scriptscriptfont1=\fivei
 \textfont2=\elevensy \scriptfont2=\sevensy \scriptscriptfont2=\fivesy
 \textfont3=\tenex \scriptfont3=\tenex \scriptscriptfont3=\tenex
 \textfont\itfam=\elevenit \def\it{\fam\itfam\elevenit}%
 \textfont\slfam=\elevensl \def\sl{\fam\slfam\elevensl}%
 \textfont\bffam=\elevenbf \def\bf{\fam\bffam\elevenbf}%
 \textfont\ttfam=\eleventt \def\tt{\fam\ttfam\eleventt}%
 \normalbaselineskip 13\p@
 \setbox\strutbox=\hbox{\vrule height9.5\p@ depth3.5\p@ width0\p@}%
 \normalbaselines\rm}
%
% restricted code size is 10 point
%
\def\rcodetype{%
 \def\rm{\fam0\tenrm}%
 \textfont0=\tenrm \scriptfont0=\sevenrm \scriptscriptfont0=\fiverm
 \textfont1=\teni  \scriptfont1=\seveni  \scriptscriptfont1=\fivei
 \textfont2=\tensy \scriptfont2=\sevensy \scriptscriptfont2=\fivesy
 \textfont3=\tenex \scriptfont3=\tenex \scriptscriptfont3=\tenex
 \textfont\itfam=\tenit \def\it{\fam\itfam\tenit}%
 \textfont\slfam=\tensl \def\sl{\fam\slfam\tensl}%
 \textfont\bffam=\tenbf \def\bf{\fam\bffam\tenbf}%
 \textfont\ttfam=\tentt \def\tt{\fam\ttfam\tentt}%
 \normalbaselineskip 13\p@
 \setbox\strutbox=\hbox{\vrule height9.5\p@ depth3.5\p@ width0\p@}%
 \normalbaselines\rm}
%
% try using 11pt for text also
%
\let\normaltype\codetype
\let\smalltype\rcodetype
\normaltype
%
% allow underscore as char in non-math text
\def\oldun{_}
\def\specialunderscore{\ifmmode_\else{\tt\char"5F}\fi}
\catcode`\_=\active
\let_\specialunderscore
%
\def\\{{\tt\char"5C}}   % for \ in comments
\def\section#1{\mark{#1}%
  {\let\folio=\relax\let_=\oldun \edef\\{\write\cfile{\endgraf
     \tocnormaltype#1\cdots\cpageno\endgraf}}\\}%
  {\bf#1.\ }}
\def\subsection#1{%
  {\let\folio=\relax\let_=\oldun \edef\\{\write\cfile{\endgraf
     \toccodetype\hskip12pt#1\cdots\cpageno\endgraf}}\\}%
  {#1.\ }}
\def\subsubsection#1{%
  {\let\folio=\relax\let_=\oldun \edef\\{\write\cfile{\endgraf
     \tocrcodetype\hskip18pt#1\cdots\cpageno\endgraf}}\\}%
  {#1.\ }}
%
%
% headline contains (title, page number, section)
%
\def\makeheadline{\vbox to0pt{\vskip-30pt
  \ifnum\pageno=-1\else\line{\strut\normaltype
      \ifodd\pageno\firstmark\hfill\bf\the\job\rlap{\ -- \folio}%
      \else\bf\llap{\folio\ -- }\the\job\rm\hfill\firstmark\fi}\fi
  \vss}\nointerlineskip}
\def\makefootline{\line{\tenrm\vrule height30pt depth0pt width0pt
   \ifodd\pageno\today\ @\rightnow\ by \cnoweb\hfill
      \else\hfill\today\ @\rightnow\ by \cnoweb\fi}}
%
\def\plainoutput{\shipout\vbox{%
  \ifodd\pageno\moveright\oddsideshift\else\moveright\evensideshift\fi
  \vbox{\makeheadline\pagebody\makefootline}}%
  \advancepageno
  \ifnum\outputpenalty>-\@MM \else\dosupereject\fi}

\let\oldpagebody\pagebody
\def\toprules{\vrule height4pt width.4pt\kern-.4pt\hskip\blockindent}
\def\pagebody{\ifnum\pageno>0\vbox to0pt{\hrule height.3pt width\hsize
      \leftline{\leaders\hbox{\toprules}}%
      \vss}\nointerlineskip
   \hbox{\llap{\vrule height\vsize width.3pt\hskip3pt}%
     \oldpagebody}%
   \else\oldpagebody\fi}

%
% table of contents
%
\newwrite\cfile
\immediate\openout\cfile=\jobname.toc
\def\cdots{\nobreak\leaders\hbox to1pc{\hss.\hss}
  \hskip24pt plus1fill}
\def\cpageno{\rlap{\hbox to18pt{\hfill\folio}}}
%
% these set the type size in toc entries
%
\let\tocnormaltype\relax   % de-activate
\let\toccodetype\relax
\let\tocrcodetype\relax
%
%
%-------------------------------------------------------------------
%
%
% All comment fields are formatted normally (plain TeX style).
% All c code is scanned and formatted verbatim.
%
% '/*' triggers TeX mode, '*/' triggers verbatim mode
%
\let\star=*   % a catcode 11 star
\let\slash=/  % a catcode 11 slash
\let\rarrow=> % a catcode 11 right arrow
%
% ---------------------------------------------------------------
%
% In TeX mode the * is active  and looks for */ or *<
% \setex sets TeX mode
%
\newif\ifrestricted\restrictedfalse
\def\setex{% begins TeX mode
  \ifrestricted\rcodetype
     \else\ifvmode\normaltype\filbreak\else\codetype\fi\fi
  \catcode`\*=\active
  {\tt/* }\ignorespaces}
%
{\catcode`\*=\active
 \gdef*{\futurelet\next\checkslash}
 }
\def\checkslash{%
    \ifx\next/\ifrestricted\let\next\resetrcc\else\let\next\setcc\fi
   \else\ifx\next<\let\next\setrcc
     \else\let\next\star\fi\fi\next}
%
%
% In code mode / is active and looks for /*
% \setcc sets the active code mode
%
\def\setcc/{\@setcc}
\def\@setcc{% set code mode
  \begingroup\ccverbatim\catcode`\/=\active
  */
  \parskip=0pt}
%
{\catcode`\/=\active
 \gdef/{\futurelet\next\checkstar}
 }
\def\checkstar{\ifx\next*\let\next\endcc
     \else\let\next\slash\fi\next}
\def\endcc*{\@endcc}
\def\@endcc{\endgroup\ifvmode\filbreak\fi\setex}
%
%
% In restricted code mode > is active and looks for >*
% \setrcc sets the restricted code mode
%
\def\setrcc<{% set little code mode
  \begingroup\restrictedtrue
  \begingroup\ccverbatim\rccverbatim\tt\catcode`\>=\active
  \catcode`\/=\active
  \parskip=0pt}
%
\def\resetrcc/{% reset little code mode
  \begingroup\ccverbatim\rccverbatim\tt\catcode`\>=\active
  \catcode`\/=\active
  */
  \parskip=0pt}
%
{\catcode`\>=\active
 \gdef>{\futurelet\next\checkrstar}
 }
\def\checkrstar{\ifx\next*\let\next\endrcc
     \else\let\next\rarrow\fi\next}
\def\endrcc*{\endgroup\endgroup}
%
%
% ---------------------------------------------------------------
%
% \leftskip is controlled by the presence of '{', '}',
%    '(', and ')' in the code text.
%
% \Ident increases \leftskip and takes effect after the current line
% \Odent decreases \leftskip and takes effect on the current line
% \IIdent and \OOdent do the same but use the long indentation
%
\newdimen\leftadjust % leftskip adjustment at next par
\def\Ident{\global\advance\leftadjust\blockindent}
\def\Odent{\global\advance\leftadjust-\blockindent\doOdent}
\def\IIdent{\global\advance\leftadjust\longblockindent}
\def\OOdent{\global\advance\leftadjust-\longblockindent\doOdent}
\def\doOdent{%
  \ifvmode\ifdim\leftadjust<\z@\global\advance\leftskip\leftadjust
     \global\leftadjust0pt\fi\fi}
%
\def\codepar{% this is \par in code mode
   \ifvmode\bigskip % this way blank lines are ignored at page tops
   \else\endgraf
     \ifdim\leftadjust=\z@\else\global\advance\leftskip\leftadjust
       \global\leftadjust0pt\fi\fi}
%
% ---------------------------------------------------------------
%
% Here are the catcodes and definitions for c-code verbatim
%
\chardef\other=12
\def\ccverbatim{%  set all chars to type `other' or `active'
  \catcode`\\=\other
  \catcode`\{=\active
  \catcode`\}=\active
  \catcode`\(=\active
  \catcode`\)=\active
  \catcode`\$=\other
  \catcode`\*=\other
  \catcode`\&=\other
  \catcode`\#=\active
  \catcode`\%=\other
  \catcode`\~=\other
  \catcode`\_=\other
  \catcode`\^=\other
  \catcode`\|=\other
  \catcode`\`=\active
  \catcode`\'=\active
  \catcode`\"=\active
  \let\par\codepar
  \obeyspaces \obeylines \frenchspacing \codetype
  \pretolerance10000 % don't hyphenate variable names, etc.
  \everypar{\hangindent\longblockindent\hangafter1}% indent wrapped 
  \rightskip0pt minus\rightmarginslop % allow some extension
  \tt}
\def\rccverbatim{% additional \other chars for restricted mode
  \catcode`\{=\other
  \catcode`\}=\other
  \catcode`\(=\other
  \catcode`\)=\other
  \catcode`\#=\other
  \catcode`\'=\other
  \catcode`\"=\other
  \rcodetype \tt}
%
% And here are the definitions for the c-code active characters.
% These perform the little formatting we do on c-code.
%     {, }, (, ), and # adjust the indentation
%     `, ', ", and / trigger more complete verbatim
%     
% Note that the right quote already has a plain-tex definition
%
{ \catcode`\<=1\catcode`\>=2
  \catcode`\{=\active \gdef{<<\Ident\char`\{>>
  \catcode`\}=\active \gdef}<<\Odent\char`\}>>
  \catcode`\(=\active \gdef(<<\IIdent\char`\(>>
  \catcode`\)=\active \gdef)<<\OOdent\char`\)>>
  \catcode`\|=0
  \catcode`\\=\active
  |catcode`|"=|active
  |gdef"<|char`|"|bgroup
     |catcode`|{=|other |catcode`|}=|other
     |catcode`|(=|other |catcode`|)=|other
     |catcode`|'=|other |catcode`|/=|other
     |catcode`|\=|active |let\|verbslash
     |def"<<|char`|"|egroup>>>
  |catcode`|'=|active
  |global|let|plain@rquote'
  |gdef|cnoweb@rquote<|char`|'|bgroup
     |catcode`|{=|other |catcode`|}=|other
     |catcode`|(=|other |catcode`|)=|other
     |catcode`|"=|other |catcode`|/=|other
     |catcode`|\=|active |let\|verbslash
     |def'<<|char`|'|egroup>>>
  |gdef'<|ifmmode|let|next|plain@rquote
     |else|let|next|cnoweb@rquote|fi|next>
 >
\def\verbslash{\begingroup\catcode`\'=\other\catcode`\"=\other\doverbslash}
\def\doverbslash#1{\\#1\endgroup}
{\catcode`\`=\active\gdef`{\relax\lq}}
{\catcode`\#=\active\gdef#{\ifvmode\ifdim\leftskip>\z@
   \noindent\llap{\hbox to\leftskip{\#\hss}}\ignorespaces
   \else\#\fi\else\#\fi}}
%
%
% ---------------------------------------------------------------
%
% \listfile produces a verbatim listing of a file
% \includefile formats the file (assumed to be c-code)
%
{\obeyspaces\gdef\hardspaces{\global\let =\ }}
\def\listfile#1{\par\begingroup\leftskip\blockindent\parskip0pt
    \ccverbatim\rccverbatim\hardspaces\input#1\par\endgroup}
\def\includefile#1{\@setcc\par\input#1\par\@endcc}
%
% ---------------------------------------------------------------
%
% \endc signals end of the c program
%
\def\endc{{\it end~\tt*/}\vfill\eject\dotitle\docontents\end}
%
% ---------------------------------------------------------------
%
% /* \newpage */ should give nice new page
\def\newpage{\vadjust{\vfill\eject}}
%
% some ebcdic systems get 05 for tabs
\catcode`\^^E=10  % make them spaces
%
% \"bold text" and \|typewriter text| and \'italic text'
%
\def\"#1"{{\bf#1}}
\def\|#1|{{\tt#1}}
\def\'#1'{{\it#1}}
%
% fix \item, etc. to work without \parindent
%
\newdimen\itemindent\itemindent2\blockindent
\def\hang{\hangindent\itemindent}
\def\textindent#1{\indent\hbox to\itemindent{\hss#1\enspace}\ignorespaces}
\def\item{\par\hang\textindent}
\def\textiindent#1{\indent\hbox to2\itemindent{\hss#1\enspace}\ignorespaces}
\def\itemitem{\par\indent\hangindent2\itemindent\textiindent}
\def\narrower{\advance\leftskip\itemindent
  \advance\rightskip\itemindent}
%
% c-no-web name
%
\def\cnoweb{\hbox{c-\setbox0=\hbox{web}%
  \vrule depth-.25\ht0 height.33\ht0 width\wd0\kern-\wd0\box0}}
%
% print the title and table of contents
%
\newtoks\job\job={\jobname}
\newtoks\title\title={}
\newtoks\synopsis\synopsis={}
\def\dotitle{
  \ifodd\pageno\else
     \message{page alignment}Intentionally blank \vfill\eject\fi
  \pageno=-1
  \vglue2pc
  \centerline{\titles\the\job}
  \vskip18pt
  \centerline{\bf\the\title}
  {\openup3pt\narrower\noindent\the\synopsis\par}
  \vfill}
\def\docontents{% print table of contents
  \mark{Contents}
  \immediate\closeout\cfile
  \begingroup
  \catcode`\@=11
  \leftskip=18pt\rightskip=\leftskip
  \everypar={\hang}
  \rightline{Page}
  \medskip
  \let\tocnormaltype\normaltype   % activate these type size macros
  \let\toccodetype\codetype
  \let\tocrcodetype\rcodetype
  \input \jobname.toc
  \catcode`\@=12
  \endgroup
  \vfill
  \eject}
\def\today{\ifcase\month\or
  January\or February\or March\or April\or May\or June\or
  July\or August\or September\or October\or November\or December\fi
  \space\number\day, \number\year}
\def\today{\ifcase\month\or
  Jan\or Feb\or Mar\or Apr\or May\or Jun\or
  Jul\or Aug\or Sep\or Oct\or Nov\or Dec\fi
  \space\number\day, \number\year}
\def\rightnow{\count0=\time\count1=\count0  \divide\count0by60
  \ifnum\count0<10 0\fi\number\count0:% hours
  \multiply\count0by60 \advance\count1by-\count0
  \ifnum\count1<10 0\fi\number\count1% minutes
  }
%
\catcode`\@=11
%
{\hsize\maxdimen
  \output={\setbox0=\box255}\eject} % delete all before \input cnoweb
\setex  % begin with TeX mode

% end of cnoweb.tex
